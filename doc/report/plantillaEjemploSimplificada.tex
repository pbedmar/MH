\input{preambuloSimple.tex}


%----------------------------------------------------------------------------------------
%	TÍTULO Y DATOS DEL ALUMNO
%----------------------------------------------------------------------------------------

\title{	
\normalfont \normalsize 
\huge{\textbf{Metaheurísticas (Curso 2021-2022)} \newline \newline Grado en Ingeniería Informática \\ Universidad de Granada} \\ [23pt] % Your university, school and/or department name(s)

\begin{figure}[H] %con el [H] le obligamos a situar aquí la figura
    \centering
        \includegraphics[scale=0.4]{img/ugr.png}
\end{figure}

\horrule{0.5pt} \\[0.4cm] % Thin top horizontal rule
\huge Práctica 1: Técnicas de Búsqueda Local y Algoritmos Greedy \linebreak \linebreak% The assignment title
\LARGE Problema a: Mínima Dispersión Diferencial
\horrule{2pt} \\[0.5cm] % Thick bottom horizontal rule
\vspace{0.7cm}

\Large{Pedro Bedmar López - 75935296Z}
\Large{pedrobedmar@correo.ugr.es} \linebreak

\large Grupo de prácticas 3 - Martes 17:30-19:30

}

\date{}

%----------------------------------------------------------------------------------------
% DOCUMENTO
%----------------------------------------------------------------------------------------

\begin{document}

\clearpage
\maketitle % Muestra el Título
\thispagestyle{empty}

\newpage %inserta un salto de página

\tableofcontents % para generar el índice de contenidos

\newpage



%----------------------------------------------------------------------------------------
%	Cuestión 1
%----------------------------------------------------------------------------------------

\part{Formulación del problema}

El problema de la \textbf{Mínima Dispersión Diferencial (MDD)} es un problema en el que dados 


\part{Descripción de la aplicación de los algoritmos}



\section{Algoritmo Greedy}


\section{Búsqueda Local}





\part{Pseudocódigo de los algoritmos}



\section{Algoritmo Greedy}


\section{Búsqueda Local}





\part{Pseudocódigo de los algoritmos}

%------------------------------------------------
\newpage

\nocite{*}

\bibliography{citas} %archivo citas.bib que contiene las entradas 

\bibliographystyle{plain} % hay varias formas de citar

\end{document}
